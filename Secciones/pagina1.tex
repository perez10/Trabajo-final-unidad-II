\section{CUESTIONARIO LABORATORIO  N° 5} 

\begin{enumerate}[1.]
	\item  Los valores introducidos al archivo sysctl.conf ¿que representan?\\ 
	fs.suiddumpable:Esto controla si el núcleo puede ser volcado desde un programa setuid como se describe anteriormente. Vea abajo. Este es un sintonizador de kernel.
	a idea es que, si hay volcados de memoria y un usuario regular puede leerlos, es posible que encuentren información privilegiada. Si el programa se descarga bien, tenía información privilegiada en la memoria, y el usuario puede leer el volcado, pueden encontrar esa información privilegiada.\\
	\\ fs.aio-max-nr:Define el máximo número de eventos permitidos en todos los contextos asíncronos de E/S. El valor predeterminado es 65536. Observe que al cambiar este valor no se preasigna o redimensiona ninguna estructura de datos de kernel.\\
	\\ fs.file-max:
	Lista el número máximo de identificadores de archivos asignados por el kernel. El valor predeterminado coincide con el valor de files-stat.max-files en el kernel, el cual se establece al valor más grande, ya sea de (mempages * (PAGE-SIZE / 1024)) / 10, o NR-FILE (8192 en Red Hat Enterprise Linux). El aumento de este valor puede corregir errores ocasionados por la falta de identificadores de archivos disponibles.\\
	\\ kernel.shmmni:Para hacer un cambio permanente, agregue la siguiente línea al archivo /etc/sysctl.conf(su configuración puede variar). Este archivo se utiliza durante el proceso de arranque.\\
	\\ kernel.sem:256 * <tamaño de la RAM en GB>\\
	\\ net.ipv4.ip-local-port-range:En Linux, hay un parámetro sysctl llamado ip-local-port-rangeque define el puerto mínimo y máximo que una conexión de red puede usar como su puerto de origen (local). Esto se aplica a las conexiones TCP y UDP.\\
	\\ net.core.rmem-default:Oracle recomienda que el tamaño predeterminado y máximo del búfer de envío ( SO-SNDBUFopción de socket) y el tamaño del búfer de recepción ( SO-RCVBUFopción de socket) se establezcan en 256 KB. TCP y UDP utilizan los buffers de recepción para retener los datos recibidos hasta que la aplicación los lea. El búfer de recepción no puede desbordarse porque el par no tiene permiso para enviar datos más allá de la ventana del tamaño del búfer. Esto significa que los datagramas se descartarán si no caben en el búfer de recepción de socket. Esto podría hacer que el remitente abrume al receptor.\\
	\\ net.core.rmem-max:Esto establece el tamaño máximo del búfer de recepción del SO para todos los tipos de conexiones\\
	\\ net.core.wmem-default:Esto establece el tamaño predeterminado del búfer de recepción del sistema operativo para todos los tipos de conexiones\\
	\\net.core.wmem-max:
	Esto establece el tamaño máximo del búfer de envío del sistema operativo para todos los tipos de conexiones.\\
	\\


\item ¿Con qué usuario(s) puedo conectarme al servidor a través del Administrador Empresarial? \\

ServiceAdministrator
Miembro de administrador de servicio de BI, autor de modelo de datos de BI y autor de carga de datos de BI. Permite a los usuarios administrar Oracle Analytics Cloud y delegar los privilegios a otros usuarios.\\
Este rol de aplicación de Oracle Identity Cloud Service se asigna automáticamente al usuario que crea el servicio.
ServiceUser
Miembro de autor de contenido de BI y autor de contenido de DV.
Permite a los usuarios crear y compartir contenido.\\
ServiceViewer
Miembro de consumidor de BI y consumidor de DV.Permite a los usuarios visualizar y explorar contenido.\\
ServiceDeployer
No se utiliza en Oracle Analytics Cloud.\\
ServiceDeveloper
No se utiliza en Oracle Analytics Cloud.\\
PODManager
Miembro del rol de administrador del servidor WebLogic global.
Permite a los usuarios crear y gestionar servicios.
\item Capture una imagen de pantalla del navegador con el Administrador Empresarial, con el nombre de su servidor e iniciada la sesión del usuario SYS. \\
\includegraphics[width=17cm]{./Imagenes/sys}

    



\end{enumerate} 