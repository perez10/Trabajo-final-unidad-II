
\section{Conclusiones} 
	\item - El sistema operativo CentOS es una distribución que si se logra utilizar perfectamente, puede llegar a ser una herramienta muy poderosa al igual que cualquier otro medio sea o no sea tangible. Este material no presenta contenido muy completo por lo cual no lleva a fondo el tema del sistema operativo CentOS, este material solo tiene la finalidad de inducir a conocer un poco sobre el manejo y uso del sistema operativo a través de comandos el cual puede ser de mucha ayuda por el simple hecho de que su contenido es muy sencillo y de fácil comprensión gracias a la explicación breve que acompaña a cada uno de los temas conforme a CentOS. En el mundo de la informática CentOS tiene un papel importante en especial, pero, no por eso lo convierte en un sistema superior a los demás sistemas que hay dentro y fuera del mercado, ya que todos buscan y tienen el mismo propósito que es el de facilitar y mejorar la forma de manejar estos sistemas, los cuales CentOS tiene una gran agilidad en estos casos para su trabajo.
\\

\section{Recomendaciones} 
	\item - Se recomienda hacer uso de un sistema de software libre  y que son los más usados en la actualidad en las empresas.
\\

\section{Bibliografía} 
\begin{itemize}
\\- http://asorufps.wikispaces.com/CENTOS.
\\- http://es.wikipedia.org/wiki/Debian.
\\- http://es.kioskea.net/contents/309-linux-estructura-de-arbol-de-los-archivos.
\\- http://www.linuxtotal.com.mx/index.php?cont=info_admon_020.
\\- Leer más: http://www.monografias.com/trabajos109/sistema-operativos-centos/sistema-operativos-centos.shtml#conclusioa#ixzz5Ab8CZ13B.

\end{itemize} 
